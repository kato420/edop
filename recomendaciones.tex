\section{Recomendaciones}

\begin{itemize}
    \item \textbf{Implementación Práctica:} Se recomienda la implementación práctica del diseño propuesto en cargadores de baterías comerciales para vehículos eléctricos. Esta implementación permitirá validar los resultados obtenidos y ajustar el diseño según las condiciones reales de operación, garantizando así que los filtros diseñados sean efectivos para mitigar los armónicos y mejorar la eficiencia energética en aplicaciones reales.

    \item \textbf{Investigación Adicional:} Se sugiere llevar a cabo investigaciones adicionales para explorar otros métodos de filtrado y tecnologías emergentes que puedan ser más eficientes y rentables. En particular, se podría investigar el uso de filtros activos o tecnologías basadas en materiales avanzados para la mitigación de armónicos.

    \item \textbf{Capacitación y Difusión:} Es importante capacitar a los ingenieros y técnicos en el uso de la función de transferencia y otras herramientas analíticas que son esenciales para el diseño y análisis de sistemas electrónicos. Además, se debe difundir los hallazgos de este estudio dentro de la comunidad académica y profesional para promover el intercambio de conocimientos y el desarrollo de soluciones innovadoras en el campo de la electrónica de potencia.

    \item \textbf{Optimización del Diseño:} Tras realizar la función de transferencia y estudiar la carga de baterías, se puede investigar un diseño de circuito más eficiente, que reduzca el ruido y sea capaz de manejar mayores tensiones sin comprometer la eficiencia del sistema.

    \item \textbf{Desarrollo de Circuitos Avanzados:} Se recomienda aprender más sobre el desarrollo de circuitos para mejorar la disección de mallas y facilitar la implementación de convertidores boost. También es importante explorar la posibilidad de utilizar más de dos convertidores boost en cascada, lo que podría mejorar aún más la eficiencia del control del factor de potencia (PFC) en sistemas más complejos.
\end{itemize}

\section*{Implicaciones}

\subsection*{Sostenibilidad:}
La implementación de filtros electrónicos eficientes en los cargadores de baterías puede contribuir significativamente a la sostenibilidad de los vehículos eléctricos, reduciendo la necesidad de reemplazos frecuentes de baterías y minimizando la generación de residuos electrónicos. Esto favorece una reducción del impacto ambiental en el sector de la automoción eléctrica.

\subsection*{Salud Pública:}
Al mejorar la eficiencia de los vehículos eléctricos y reducir su impacto ambiental, se contribuye a la mejora de la calidad del aire, lo que tiene un impacto positivo en la salud pública. La reducción de las emisiones contaminantes puede reducir problemas respiratorios y cardiovasculares relacionados con la contaminación.

\subsection*{Desarrollo Tecnológico:}
Este estudio subraya la importancia de la investigación continua en el campo de la electrónica y el diseño de sistemas. Promueve el desarrollo de tecnologías más avanzadas y eficientes, lo que podría generar nuevas soluciones que mejoren la eficiencia energética y la estabilidad en sistemas eléctricos, tanto en vehículos eléctricos como en otras aplicaciones de energía renovable y electrónica de potencia.