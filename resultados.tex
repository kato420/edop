\section{Resultados}

\subsection{Introducción a los resultados}
En este proyecto, se ha analizado el comportamiento del sistema de carga de baterías de vehículos eléctricos utilizando un convertidor boost. El objetivo principal ha sido modelar y simular el comportamiento dinámico del sistema mediante la resolución de ecuaciones diferenciales que describen el proceso de carga, tanto en su fase transitoria como estacionaria. A lo largo del análisis, se emplearon métodos analíticos para obtener soluciones exactas en sistemas simples y métodos numéricos para abordar situaciones más complejas. Los resultados obtenidos permiten evaluar la efectividad de los filtros diseñados para mitigar los armónicos en el proceso de carga, así como la influencia de los parámetros del sistema en el rendimiento general del cargador. En esta sección, se presentan los resultados obtenidos de las simulaciones y las soluciones analíticas, comparándolos con datos experimentales y soluciones teóricas previas.

\subsection{Presentación estructurada de los hallazgos}
% Espacio reservado para contenido futuro

\subsection{Uso de figuras y tablas}
% Espacio reservado para contenido futuro

\subsection{Comparaciones}
% Espacio reservado para contenido futuro

\subsection{Resumen de resultados}
% Espacio reservado para contenido futuro