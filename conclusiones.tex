\section{Conclusiones}

\subsection{Resumen de los hallazgos principales:}
Este estudio ha logrado optimizar el diseño y análisis de filtros electrónicos para mitigar los armónicos en los sistemas de carga de baterías de vehículos eléctricos. Se ha demostrado que los armónicos presentes en el sistema son una fuente significativa de ineficiencia y posibles daños a las baterías. El diseño propuesto, que incluye convertidores boost en cascada, ha mostrado una mejora notable tanto en la corrección del factor de potencia (PFC) como en la regulación de la corriente inyectada a las baterías. Además, el análisis mediante herramientas analíticas, como el uso de mallas y la ley de Kirchhoff en el circuito RLC del cargador, ha permitido una comprensión profunda del comportamiento del sistema y ha optimizado el diseño del filtro.

\subsection{Relevancia teórica y práctica de los resultados:}
Los resultados obtenidos son relevantes tanto desde el punto de vista teórico como práctico. En términos teóricos, la aplicación de la función de transferencia en el diseño de filtros electrónicos ha sido validada como una herramienta eficaz para mitigar los armónicos, contribuyendo al entendimiento de cómo la teoría se traduce en soluciones prácticas para los sistemas de carga de vehículos eléctricos. Desde el punto de vista práctico, estos resultados tienen una implicación directa en la mejora de la eficiencia energética de los cargadores de baterías, al reducir los armónicos que afectan el proceso de carga y la durabilidad de las baterías.

\subsection{Relevancia para el campo de estudio:}
Este estudio resulta crucial para el campo de la electrónica de potencia y el diseño de
cargadores de baterías para vehículos eléctricos. La optimización de los filtros electrónicos a
través de la reducción de armónicos no sólo mejora la eficiencia del proceso de carga, sino
que también minimiza los daños potenciales a las baterías. Estos resultados abren la puerta a
nuevas investigaciones y aplicaciones en el diseño de cargadores más eficientes y en el
desarrollo de tecnologías que integren modelos más complejos y avanzados.

\subsection{Síntesis y relevancia del estudio:}
En conclusión, este estudio proporciona una base sólida para el diseño y la optimización de
filtros electrónicos en sistemas de carga de baterías de vehículos eléctricos. Los resultados
obtenidos son de gran valor tanto para la teoría como para la práctica, ya que permiten
mejorar la eficiencia energética, mitigar los armónicos y optimizar el proceso de carga. Este
trabajo representa un paso importante hacia el desarrollo de cargadores más eficientes y
sostenibles para vehículos eléctricos, con implicaciones significativas en la mejora de la
infraestructura de carga y la eficiencia energética en general.