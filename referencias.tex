\section{Referencias}
\renewcommand{\refname}{}  % Esto borra el título automático
\begin{thebibliography}{9}

    \bibitem{iea2023clean}
    Agencia Internacional de Energía. (2023). \textit{Electric vehicles and the transition to clean energy}. \url{https://www.iea.org/energy-system/transport/electric-vehicles}

    \bibitem{iea2023outlook}
    Agencia Internacional de Energía. (2023). \textit{Global EV Outlook 2023}. \url{https://www.iea.org/reports/global-ev-outlook-2023}

    \bibitem{abundis2016}
    Abundis, A. (2016). \textit{Causas y efectos de armónicos en sistemas eléctricos de potencia}. Universidad Nacional Autónoma de México, Facultad de Ingeniería. \url{http://132.248.52.100:8080/xmlui/handle/132.248.52.100/11159}

    \bibitem{arias2015}
    Arias, D. (2015). \textit{Influencia del vehículo eléctrico sobre la fiabilidad de los sistemas eléctricos}. Escuela Politécnica Superior, Grado de Ingeniería en Tecnologías Industriales. \url{https://hdl.handle.net/10016/23428}

    \bibitem{bolanos}
    Bolaños, C. V. J. (s/f). \textit{Modelado de sistemas eléctricos y funciones de transferencia}. \url{https://suayed.cuautitlan.unam.mx/uapas/01_ModSisEle_FuncDeTrans/}

    \bibitem{david2021}
    David, V. (2021). \textit{Modelado de sistemas eléctricos y funciones de transferencia}. Universidad Nacional Autónoma de México.

    \bibitem{greenpeace2010}
    Greenpeace. (2010). \textit{El transporte y las emisiones de gases de efecto invernadero}. Recuperado de \url{https://archivo-es.greenpeace.org/espana/Global/espana/report/other/2010-10-26-2.pdf}

    \bibitem{lazaro2016}
    Lázaro, H., Melgarejo, G., Montoro, E., Obregón, J., Diego, J., \& Aramburú, V. (2016). Aplicación de la transformada de Laplace a circuitos eléctricos. \textit{Revista Del Instituto De investigación De La Facultad De Minas, Metalurgia Y Ciencias geográficas}, 19(38), 43-46. \url{https://doi.org/10.15381/iigeo.v19i38.13566}

    \bibitem{orellana2022}
    Orellana Uguña, C. M., González Morales, L., \& Verdugo, K. (2022). Diseño de un cargador rápido de baterías para vehículos eléctricos enchufables en el punto de conexión común de la red de distribución de energía eléctrica. \textit{Elektron}, 6(2), 77-85. \url{https://doi.org/10.37537/rev.elektron.6.2.161.2022}

    \bibitem{paipa2020}
    Paipa, César C., Ramirez, Julio C., Trujillo R., César L., Alarcón V., Jorge A., \& Jaramillo M., Adolfo A. (2020). Battery charger design with low current harmonic distortion for application in electric vehicles. \textit{Ingeniare. Revista chilena de ingeniería}, 28(4), 706-717. \url{https://dx.doi.org/10.4067/S0718-33052020000400706} (PRINCIPAL)

    \bibitem{rogers2008}
    Rogers Acevedo, G. G. (2008). \textit{Diseño sistema de filtros de armónicas en corriente alterna para un enlace HVDC} [Tesis de pregrado, Universidad de Chile]. Repositorio Académico Universidad de Chile. \url{https://repositorio.uchile.cl/handle/2250/103154}

    \bibitem{toro2015}
    Toro Cea, M. (2015). \textit{Diseño de estrategias de control para operación desbalanceada de microrredes de baja tensión}. [Tesis de pregrado]. Repositorio Académico Universidad de Chile. \url{https://repositorio.uchile.cl/handle/2250/134595}

    \bibitem{zhang2021}
    Zhang, Y., Wang, Z., \& Li, Y. (2021). A New Feedback Method for PR Current Control of LCL-Filter-Based Grid-Tied Inverters. \textit{Energies}, 14(5), 1303. \url{https://doi.org/10.3390/en14051303}

    \bibitem{sainz2011}
    Sainz, L., \& Balcells, J. (2011). Experimental measurements about harmonic current mitigation of electric vehicle battery chargers. \textit{Renewable energy \& power quality journal}, 407-412. \url{https://doi.org/10.24084/REPQJ09.349}

    \bibitem{cittanti2021}
    Cittanti, D., Mandrile, F., \& Bojoi, R. (2021). Design Space Optimization of a Three-Phase LCL Filter for Electric Vehicle Ultra-Fast Battery Charging. \textit{Energies}, 14(5), 1303. \url{https://doi.org/10.3390/en14051303}

    \bibitem{carvajal2011}
    Carvajal Carreño, W., Ordóñez Plata, G., Moreno Wandurraga, A. L., \& Duarte Gualdrón, C. A. (2011). Simulación de sistemas eléctricos con cargas no lineales y variantes en el tiempo. \textit{Ingeniare. Revista Chilena de Ingeniería}, 19(1), 76-92. \url{http://dx.doi.org/10.4067/S0718-33052011000100009}

\end{thebibliography}