\section {Resumen}
El objetivo principal de este proyecto es optimizar respecto al diseño de filtros electrónicos en convertidores boost utilizados en cargadores de baterías para vehículos eléctricos, con la finalidad de reducir la distorsión de armónicos y mejorar la calidad de la energía entregada. Los armónicos que tienen estos sistemas afectan negativamente en la eficiencia del proceso de carga y la vida útil de las baterías. Se abordó el problema modelando el sistema eléctrico con ecuaciones diferenciales ordinarias (EDO), y se analizó el comportamiento dinámico del sistema mediante herramientas matemáticas como la transformada de Laplace y la función de transferencia. Nos basamos en datos extraídos de estudios anteriores para obtener los parámetros representativos, además de ello se implementó simulaciones numericas en Python, lo cual permitió comparar las soluciones teóricas con los resultados obtenidos computacionalmente. El análisis especial mediante la Transformada Rápida de Fourier (FFT) demostró la atenuación efectiva de armónicos superiores, validando la eficacia del filtro propuesto. En conclusión, este estudio evidencia que el uso de métodos analíticos junto con simulaciones en Python constituye una estrategia efectiva para el diseño y validación de sistemas de carga más estables, eficientes y sostenibles para vehículos eléctricos.

\textbf{Palabras clave:} armónicos, convertidor boost, función de transferencia, transformada de Laplace, vehículos eléctricos.
