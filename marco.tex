\section{Marco teórico}
En el contexto de la movilidad eléctrica, el estudio de los sistemas eléctricos ha cobrado relevancia
debido al creciente uso de vehículos eléctricos, cuyos cargadores están expuestos a perturbaciones
como los armónicos. Estas fluctuaciones en la corriente eléctrica afectan la calidad de la carga,
reduciendo la eficiencia del sistema y acortando la vida útil de componentes como las baterías. Este
marco teórico aborda los principios esenciales para dicho análisis, incluyendo las leyes de Kirchhoff,
fundamentales en el estudio de circuitos eléctricos. Para mitigar estos efectos, se emplean
herramientas matemáticas como la función de transferencia y la Transformada de Laplace, que
permiten modelar el comportamiento del sistema y diseñar filtros electrónicos adecuados (Zhang,
Wang, \& Li, 202, pp.4-5).

\subsection{ Conceptos}

\subsubsection{ Función de Transferencia:}
La función de transferencia es una herramienta matemática utilizada para describir la
relación entre la entrada y la salida de un sistema dinámico en el dominio de Laplace. Es
especialmente útil en la ingeniería de control y en el análisis de sistemas eléctricos, donde
permite modelar la respuesta del sistema ante perturbaciones como los armónicos. En el caso
de los cargadores de baterías de vehículos eléctricos, la función de transferencia ayuda a
modelar el comportamiento de los filtros electrónicos que mitigan los armónicos presentes en
el sistema de carga (Bolaños, s/f).

\subsubsection{ Transformada de Laplace:}
La Transformada de Laplace es una herramienta esencial en el análisis de sistemas
dinámicos, permitiendo convertir ecuaciones diferenciales en ecuaciones algebraicas. Este
enfoque simplifica la resolución de problemas complejos en circuitos eléctricos y otros
sistemas dinámicos. En el contexto de los cargadores de baterías, la transformada de Laplace
es útil para analizar cómo los filtros afectan la señal de entrada, permitiendo el diseño de
soluciones más efectivas para mitigar los armónicos (Lázaro et al., 2016).

\subsubsection{ Leyes de Kirchhoff:}
Las leyes de Kirchhoff son esenciales para la evaluación de circuitos eléctricos. Así mismo,
la ley de corrientes de Kirchhoff (LCK) establece que la suma de las corrientes en un nodo es
cero, mientras que la ley de voltajes de Kirchhoff dicta que la suma de los voltajes en una
malla cerrada es cero. Estas leyes son claves para entender el flujo de energía en los sistemas
de carga de baterías y son esenciales en el diseño de filtros que ayudan a reducir los
armónicos (Arias, 2015).
\newpage
\subsubsection{ Armónicos en Circuitos Eléctricos:}
Los armónicos son componentes de frecuencia múltiple de una señal fundamental que se
superponen a la señal de corriente o voltaje en un sistema eléctrico. Estas frecuencias
adicionales son generadas por la no linealidad de los dispositivos conectados al sistema,
como convertidores de potencia o cargadores de baterías, que generan distorsiones en la
forma de onda de la corriente o el voltaje. Estos armónicos afectan negativamente la calidad
de la señal eléctrica y pueden interferir con el funcionamiento adecuado de los componentes
del sistema (Abundis, 2016).

\subsection{ Campo de Aplicación Específico}
El campo de aplicación específico de este proyecto es el sistema eléctrico de carga de
baterías para vehículos eléctricos. Este sistema es un ejemplo de un sistema dinámico que se
ve afectado por la presencia de armónicos, los cuales son fluctuaciones en la corriente que
pueden afectar negativamente la eficiencia de la carga y la vida útil de las baterías. El estudio
se enfoca en cómo los filtros electrónicos, modelados a través de ecuaciones diferenciales y
utilizando la función de transferencia, pueden mitigar estos armónicos, mejorando así el
rendimiento del sistema de carga.

\subsubsection{ Métodos de Resolución}

\textbf{I. Análisis mediante la función de transferencia:}\\
Para resolver el problema de los armónicos en los sistemas de carga, se utilizará la
función de transferencia para modelar el comportamiento de los filtros electrónicos.
Esto permitirá predecir la respuesta del sistema ante perturbaciones y optimizar el
diseño de los filtros.

\textbf{II. Resolución mediante la Transformada de Laplace:}\\
Se aplicará la Transformada de Laplace a las ecuaciones diferenciales que describen
los circuitos eléctricos, transformándolas en ecuaciones algebraicas más fáciles de
manejar (Lázaro et al., 2016, pp. 43-46). Este método es crucial para analizar cómo
los filtros electrónicos pueden mitigar los armónicos y mejorar la eficiencia del
sistema de carga.

\textbf{III. Simulación y análisis de estabilidad:}\\
Se realizarán simulaciones del sistema de carga de baterías, modelado mediante la
función de transferencia y las ecuaciones diferenciales, para evaluar el impacto de
los armónicos y la efectividad de los filtros. Además, se analizará la estabilidad del
sistema utilizando herramientas matemáticas basadas en las leyes de Kirchhoff y
otros métodos de resolución analítica.
