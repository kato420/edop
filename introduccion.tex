% introduccion.tex
\section{Introducción}

El transporte es un componente esencial de la sociedad moderna, ya que facilita el movimiento de personas y bienes, impulsa la economía y favorece la interacción social. Con el tiempo, los vehículos se han convertido en el principal medio para satisfacer estas necesidades. Actualmente, el sector atraviesa una transformación impulsada por el auge de los vehículos eléctricos, que, al funcionar con energía eléctrica en lugar de combustibles fósiles, representan una alternativa más sostenible y contribuyen a la disminución de la contaminación ambiental. Según la Agencia Internacional de Energía (2023), la adopción de vehículos eléctricos podría reducir las emisiones de carbono en un 30\% para el 2040, lo que destaca aún más la importancia de esta transición.

A pesar de las ventajas ambientales de los vehículos eléctricos, uno de los retos más importantes radica en la gestión óptima de la energía almacenada en sus baterías. Un problema frecuente en este tipo de sistemas es la presencia de armónicos, que son fluctuaciones no deseadas en la corriente eléctrica. Estos armónicos afectan tanto el rendimiento del sistema como la vida útil de las baterías, lo que subraya la necesidad de una solución eficaz para mitigar este problema. Además de disminuir la eficiencia del sistema, los armónicos también generan pérdidas térmicas en los convertidores de los cargadores, lo que impacta directamente en su durabilidad y desempeño (Orellana Uguña et al., 2022). Esto evidencia que es necesario complementar el diseño de filtros con estrategias que aborden integralmente los efectos de los armónicos en sistemas de carga. En ese contexto, se ha documentado que ``los filtros de armónicas son esenciales para mitigar las distorsiones en sistemas HVDC, mejorando así la calidad de la energía transmitida'' (Rogers Acevedo, 2008, p. 15).

Existen estudios recientes que han explorado la reducción de armónicos en sistemas de carga mediante diversas técnicas de filtrado. Una de las soluciones más destacadas es el uso de filtros LCL, los cuales, aunque han demostrado ser eficaces para mejorar la calidad de la corriente, requieren un diseño preciso del sistema y una cuidadosa elección de sus parámetros para garantizar su efectividad (Zhang et al., 2021). Sin embargo, la mayoría de estos enfoques no han logrado optimizar de manera significativa la eficiencia de los filtros, especialmente en escenarios de carga rápida.

La función de transferencia, como herramienta matemática clave en el análisis de sistemas dinámicos, ofrece una forma eficaz de modelar y optimizar el comportamiento de estos filtros, mejorando su desempeño en diversas condiciones. Además, el uso de modelos con función de transferencia no solo potencia la capacidad predictiva del comportamiento dinámico del sistema, sino que también permite diseñar filtros adaptativos que se ajusten a distintas condiciones operativas. A esto se suma la importancia de implementar controles que compensen componentes de secuencia negativa y cero, garantizando así una operación más estable de las microrredes (Toro Cea, M., 2015, p. 22). En este proyecto, se pretende utilizar la función de transferencia en los filtros eléctricos en cargadores de baterías para vehículos eléctricos, no solamente optimizando su rendimiento, sino también en la estabilización de la corriente, lo que ayuda a evitar sobrecargas, reducir los armónicos y contribuir a un futuro más sostenible en el transporte.

\vspace{1cm}
% --- PREGUNTA DE INVESTIGACIÓN ---
\noindent\textbf{Pregunta de investigación:} \\
¿Cómo podría el uso de la función de transferencia en un filtro electrónico mitigar los armónicos en los sistemas eléctricos de carga de vehículos eléctricos?