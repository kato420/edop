\section{Justificación}
Los sistemas de carga de baterías para vehículos eléctricos presentan múltiples desafíos en su funcionamiento, siendo uno de los desafíos más importantes la presencia de armónicos. Estas perturbaciones o alteraciones afectan la calidad de la energía, reduciendo así la eficiencia del proceso de carga y acortando la vida útil de las baterías. Debido a la creciente aprobación de estos vehículos, es fundamental optimizar la calidad de los sistemas de carga, a través del diseño de soluciones que mitiguen estas distorsiones eléctricas.

Este trabajo plantea el análisis y mejora de filtros electrónicos utilizados en los cargadores, empleando herramientas del análisis de sistemas como la función de transferencia, la transformada de Laplace y las leyes de Kirchhoff. Estas permiten representar con precisión el comportamiento eléctrico del sistema, facilitando su evaluación y optimización. El objetivo es identificar configuraciones que reduzcan eficazmente los armónicos, estabilice la corriente y evitar fluctuaciones indeseadas durante la carga.

La propuesta se vincula directamente con los contenidos del curso de Ecuaciones Diferenciales Ordinarias, dado que la modelación de estos sistemas se basa en ecuaciones diferenciales lineales. La aplicación de estos métodos a un problema real permite reforzar los conceptos aprendidos y demostrar su utilidad en el diseño de soluciones técnicas en el campo de la ingeniería eléctrica.